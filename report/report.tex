\documentclass[12pt,a4paper]{article}
\usepackage{prettytex/base}
\usepackage{prettytex/math}
\usepackage{lipsum}

\setlength{\topmargin}{0.0in}
\setlength{\oddsidemargin}{0.33in}
\setlength{\textheight}{9.0in}
\setlength{\textwidth}{6.0in}
\renewcommand{\baselinestretch}{1.25}

\newcommand{\topictitle}{Solving PDEs using Spectral Methods in the Chebyshev basis \\ \large by example of the Heat Equation}
\newcommand{\candidatenumber}{12345}
\newcommand{\course}{Approximation of Functions}

\title{\topictitle}
\author{Candidate \candidatenumber}
\date{\today}

\begin{document}
  \begin{center}
    \Large \topictitle \\
    \vspace{.3cm}

    \normalsize Special Topic on \textcolor{themecolor3}{\textsc{\course}}\\
    \normalsize Candidate Number: \textcolor{themecolor3}{\candidatenumber}
    \vspace{.3cm}
  \end{center}

  \begin{abstract}
    Insert some waffle here if you want. Abstracts are optional for
    special topics but are part of page count.
  \end{abstract}

  \section{Introduction}
  \lipsum

  \section{The heat equation and its solution}
  \lipsum

  \section{Chebyshev}
  \lipsum

  \section{Differentiation}
  \lipsum

  \section{Results}
  \section{Discussion}

  \appendix

  \section{Title of Appendix}
  Appendices are definitely not necessary and assessors are not obliged to read them so only use them for non-vital text, figures or calculations.
\end{document}
