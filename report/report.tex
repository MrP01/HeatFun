\documentclass[12pt, a4paper]{article}
\PassOptionsToPackage{sharp}{prettytex/boxes}
\usepackage{prettytex/base}

\setlength{\topmargin}{0.0in}
\setlength{\oddsidemargin}{0.33in}
\setlength{\textheight}{9.0in}
\setlength{\textwidth}{6.0in}
\renewcommand{\baselinestretch}{1.25}

\usepackage{prettytex/math}
\usepackage{prettytex/math-theorems}
\usepackage{prettytex/mathematicians}
\usepackage{prettytex/gfx}
\usepackage{prettytex/thesis}
\usepackage{cleveref}
\usepackage{lipsum}

\setlength{\headheight}{19.53pt}
\setlength{\headsep}{1.8em}
\setlength{\belowcaptionskip}{-12pt}
\addbibresource{sources.bib}

\newcommand{\topictitle}{Solving PDEs using Spectral Methods in the Chebyshev basis \\ \large by example of the Heat Equation}
\newcommand{\candidatenumber}{12345}
\newcommand{\course}{Approximation of Functions}

\title{\topictitle}
\author{Candidate \candidatenumber}
\date{\today}

% Outline:
% Setting: define heat equation PDE problem (with BCs), explain series in x direction
% Introduce Chebyshev polynomials (relation of x, z and theta)
% Prove recurrence relation
% Prove orthogonality
% - Using transformation to theta
% - cite Cauchy integral for Laurent series
% Define Lipschitz
% Define Function Space in Cheb Basis
% Introduce Cheb series
% Algorithm one: interpolantThrough()
% - Method one: Rectangular integral approx. rule
% - Method two: DCT
% - Method three: Barycentric formula
% Algorithm two: evaluateOn()
% Numerics: Define Forward Euler
% Algorithm three: differentiation of a Cheb series
% - Method one: via DCT (todo...)
% - Method two: via recursion formula
% - Method three: via finite differences and repeated interpolantThrough()
% Final application: interpolantThrough(), u_1 = alpha * dt * diff(diff(u_0)), evaluateOn()
% Results

\begin{document}
  \pagestyle{plain}
  \begin{center}
    \Large \topictitle \\
    \vspace{.3cm}

    \normalsize Special Topic on \textcolor{themecolor3}{\textsc{\course}}\\
    \normalsize Candidate Number: \textcolor{themecolor3}{\candidatenumber}
    \vspace{.3cm}
  \end{center}

  \begin{abstract}
    Insert some waffle here if you want. Abstracts are optional for
    special topics but are part of page count.
  \end{abstract}

  \section{Introduction}
  Let $\N$ denote the nonnegative integers, so $0 \in \N$.

  \begin{definition}{Chebyshev polynomial}{chebpoly}
    Chebyshev\footnote{named after Pafnuty Lvovich \textsc{Chebyshev}, alternatively transliterated as Tchebycheff, Tchebyshev (French) or Tschebyschow (German)} polynomials $T_k: \R \mapsto \R$ are functions satisfying
    \begin{align*}
      T_k(x) = T_k(\cos \theta) := \cos(k \theta) = \frac{1}{2} (z^k + z^{-k}) \\
      z := \e^{i \theta},\quad x := \Re(z) = \cos(\theta) = \frac{1}{2}(z + z^{-1})
    \end{align*}
    for $k \in \N$. Then, $T_0(x) = 1$, $T_1(x) = x$, $T_2(x) = 2x^2-1$, and so on.
  \end{definition}

  These relations between $x$, $z$ and $\theta$ reveal fundamental connections between three famous basis sets (as we will confirm later): \textsc{Chebyshev}, \textsc{Legendre} and \textsc{Fourier}.

  \begin{theorem}{Chebyshev Recursion Formula}{chebrecursion}
    The Chebyshev polyomials satisfy the three-term recurrence relation $$T_{k+1}(x) = 2x T_k(x) - T_{k-1}(x) \,.$$
  \end{theorem}
  \begin{proof}{\autoref{thm:chebrecursion}}
    For $k > 1$,
    \begin{align*}
      2x T_k(x) - T_{k-1}(x) & = 2x \frac{1}{2} (z^k + z^{-k}) - \frac{1}{2} (z^{k-1} + z^{-(k-1)})                       \\
                             & = 2 \frac{1}{2}(z + z^{-1}) \frac{1}{2}(z^k + z^{-k}) - \frac{1}{2} (z^{k-1} + z^{-k+1})   \\
                             & = \frac{1}{2} (z^{k+1} + z^{k-1} + z^{-k+1} + z^{-k-1}) - \frac{1}{2} (z^{k-1} + z^{-k+1}) \\
                             & = \frac{1}{2} (z^{k+1} + z^{-(k+1)}) = T_{k+1}(x)
    \end{align*}
  \end{proof}
  \pagestyle{normal}

  The Chebyshev polynomials also satisfy an \emph{orthogonality relation},
  $$\langle T_m, T_n \rangle := \int_{-1}^1 T_m(x) T_n(x) \frac{1}{\sqrt{1-x^2}} \ddx = \int_{\pi}^{0} \cos(m \theta) \cos(n \theta) \frac{-\sin(\theta)}{\sqrt{1-\cos^2(\theta)}} \dd\theta \,,$$
  which becomes, with the fitting substitution $x = \cos(\theta)$ and $\ddx = -\sin(\theta) \dd\theta$,
  \begin{align*}
    \langle T_m, T_n \rangle & = \int_0^\pi T_m(\cos \theta) T_n(\cos \theta) \frac{\sin \theta}{\sin \theta}\dd\theta = \int_0^\pi \cos(m \theta) \cos(n \theta) \dd\theta                            \\
                             & = \frac{1}{2} \int_0^\pi \big(\underbrace{\cos((m+n) \theta)}_{=\cos(2m\theta) \text{ for } m=n} + \underbrace{\cos((m-n) \theta)}_{=1 \text{ for } m=n}\big) \dd\theta
  \end{align*}
  along with the knowledge that $\int_0^\pi \cos(k \theta) \dd\theta = k^{-1} \left[\sin(k\theta)\right]_0^\pi = 0$ for $k \in \Z \backslash \{0\}$,
  $$\langle T_m, T_n \rangle = \int_0^\pi T_m(\cos \theta) T_n(\cos \theta) \dd\theta = \begin{cases}
      0     & \text{ for } m \neq n     \\
      \pi/2 & \text{ for } m = n \neq 0 \\
      \pi   & \text{ for } m = n = 0
    \end{cases}$$
  which can be effectively utilised to define a function space $(\mathcal{T}, +, \cdot)$ in the \emph{orthogonal} basis of Chebyshev polynomials $\mathcal{T} := \{T_k\}_{k \in \N}$.
  Note that the operation $\langle \cdot, \cdot \rangle$ satisfies all axioms of an authentic inner product (linearity, etc.) over a function space due to the linearity of the integral.

  In the following proceedings, we will restrict our view on functions over the interval $[-1, 1] \subset \R$.
  Any (real) Lipschitz-continuous function $f \in \mathcal{C}_L$, where $\cC_L := \{g: [-1, 1] \mapsto \R \;|\; \exists L \text{ s.t. } \forall x_1, x_2 \in \R, \; |g(x_1) - g(x_2)| \le L \cdot |x_1-x_2|\}$ can be represented in the Chebyshev basis $\cT$, as Lipschitz continuity is a sufficient condition for absolute and uniform convergence of the corresponding series representation
  $$f(x) = \sum_{k=0}^\infty a_k T_k(x), \quad a_k \in \R,\quad k \in \N \,.$$

  Utilising orthogonality, for any $f \in \cC_L$, we find coefficients $a_l \in \R$ by 'right-multiplying' the equation $f = \sum_{k=0}^\infty a_k T_k$ with any one of the Chebyshev polynomials $T_l$.
  \begin{align*}
    \langle f, T_l \rangle & = \langle \sum_{k=0}^\infty a_k T_k, T_l \rangle = \int_0^\pi a_k T_k(\cos \theta) T_l(\cos \theta) \dd\theta \\
                           & = \sum_{k=0}^\infty a_k \langle T_k, T_l \rangle \quad \text{ \textcolor{gray}{by linearity}}                 \\
                           & = \begin{cases}
                                 a_0 \pi   & \text{ for } l = 0    \\
                                 a_l \pi/2 & \text{ for } l \neq 0
                               \end{cases}
  \end{align*}
  which can easily be rearranged to give explicit relations for $a_0$ and $a_k$
  \begin{align*}
    a_0 & = \frac{1}{\pi} \langle f, T_0 \rangle =  \frac{1}{\pi} \int_0^\pi f(\cos \theta) \dd\theta                                   \\
    a_k & = \frac{2}{\pi} \langle f, T_k \rangle = \frac{2}{\pi} \int_0^\pi f(\cos \theta) \cos(k \theta) \dd\theta, \quad k \neq 0 \,.
  \end{align*}
  Dealing with a numerical problem, we shall approximate the above two integrals by the rectangular integral rule.
  Another proof for the explicit coefficient integrals can be found in \cite{atap}.

  \begin{theorem}{Rectangular integral rule}{integralapprox}
    $$\int_a^b f(x) \ddx = \lim_{N \rightarrow \infty} \frac{b-a}{N} \sum_{k=0}^N f(x_k), \quad x_k := a + \frac{b-a}{N} k$$
  \end{theorem}
  \cite[128]{bonthuis-cp}

  Most importantly, this quadrature-style integral approximation is only one way of numerically determining the coefficients $a_k$.
  Another is to recognise the structure of the above integral for $k \neq 0$ as a cosine transform of the function $(f \circ \cos)$.

  \begin{definition}{Cosine Transform}{costrans}
  \end{definition}

  \begin{definition}{Discrete Cosine Transform}{dct}
  \end{definition}

  Most significantly, this approach via the Discrete Cosine Transform can be sped up by means of the \emph{Fast Fourier Transform} \parencite{cooley-tukey-fft}.

  Numerically speaking, a significant improvement to these two approaches can be made by using the \emph{Barycentric interpolation formula in Chebyshev points}.

  \begin{definition}{Chebyshev points}{chebpoints}
    From the equispaced points
    $$\Theta := \{\theta_j := jN/\pi \;|\; j = 0, ..., N\} \,,$$
    we can further define the Chebyshev points as the corresponding $\cos(\theta_j)$,
    $$X := \{x_j := \cos(\theta_j) \;|\; \theta_j \in \Theta\} \,.$$
  \end{definition}

  One way of forcing the boundary conditions, at least the first that came to my mind when thinking of this issue, is to pin down the two highest-order coefficients in the series representation.

  \section{The heat equation and its solution}
  \section{Differentiation}
  \section{Results}
  \begin{figure}[H]
    \centering
    \inputtikz{numerical-comparison}
    \caption{Comparison of heatfun and chebfun}
  \end{figure}

  \section{Discussion}

  \printbibliography

  \appendix
  \section{Title of Appendix}
  Appendices are definitely not necessary and assessors are not obliged to read them so only use them for non-vital text, figures or calculations.
\end{document}
